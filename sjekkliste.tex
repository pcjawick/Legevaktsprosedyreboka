\newpage
\chapter{Når legevakten begynner}\label{chap:lv_start}
	\section{Kontoret, dataen og hvilerommet}\label{sec:start_kontor}
	Denne listen har 13 sjekkpunkter som må gjennomføres før legevakten begynner og sjekkes med sykepleier. Se side \pageref{chap:desc_lv_start} hvis du lurer på hvorfor. \\
		\begin{itemize}
			%kommentert litt her for å gjøre lista kortere
			%\item PC
				%\begin{itemize}
					\item Sett inn BuyPasskortet før du starter Citrix
					\item Brukernavn og passord ligger i skuffen på venstre  
				%\end{itemize}
			\item Sjekk om det er nok papir i skriveren
			%\item Jeg har sjekket telefonlista(oppdateres forløpende, må sjekkes hver gang) bakerst i boka slik at jeg vet hvilke nummer som er viktige
			\item Hvis jeg tar i mot penger husker jeg å trykke "alt betalt"
		\end{itemize}
	\section{Sykepleier}\label{sec:start_sykepleier}
		\begin{itemize}
			\item Presenter deg for sykepleieren
			\item Avtal minst en matpause
			\item Avtal hvor lenge du trenger per pasient
		\end{itemize}
	\section{Nødnett og AMK}\label{sec:start_AMK}
		\begin{itemize}
			\item Skru av tastelås, rykk og hold '3' inn og og sjekk at du ringer opp legevakta
			\item Trykk og hold '2' inne, deretter bruk den store sorte knappen, trykk og hold inne og fortell hvem du er og at du er på vakt i Øvre Eiker.
			\item Trykk og hold '1' inne til linja i midten viser 'Øvre Eiker legevakt'
		\end{itemize}
	\section{Samarbeid før vakt}\label{sec:start_samarbeid}
		\subsection{Legevakta}\label{sec:sam_lvintern}
			\begin{itemize}
				\item Sjekket om det ligger aktuelle beskjeder i 'beskjeder' fil som ligger på skrivebordet
			\end{itemize}
		\subsection{Politiet}\label{sec:sam_pol}%Vurderer dette på sykepleiersjekkliste
			\begin{itemize}
				\item Ring 02800 og sjekk hvilket lensmannskontor som har vakt i dag
			\end{itemize}
		\subsection{Hjemmetjenesten}\label{sec:sam_hjtj}%Vurderer dette på sykepleiersjekkliste
			\begin{itemize}
				\item Be sykepleier ringe hjemmetjenesten og spørre om det er noen som skal ha sykebesøk i dag
		\end{itemize}
		\label{lst:start_kontor_num}
\newpage
\chapter{Etter legevakta}
	\section{Regnskap}\label{sec:elv_regns}
		\begin{itemize}
			\item Sendt samleregning til HELFO
			\item Skrevet ut kassadagbok til regnskapet
			\item Sjekket http://portal.credicare.no/ at jeg ikke har noen ting utestående
		\end{itemize}
	\section{Neste vakt}\label{sec:elv_nxtv}
		\begin{itemize}
			\item Hvilken dato er neste vakt? Passer det?
		\end{itemize}
	\section{Tilbakemeldinger}\label{sec:elv_feedbk}
		\begin{itemize}
			\item SMS til \pawmob{} om det er noe som må endres på legevakta, eller noe som ikke har fungert.
		\end{itemize}

\newpage
\chapter{Medisinske huskelister}
	\section{Akuttmedisin}
		\subsection{A-HLR}
			\begin{itemize}
				\item
			\end{itemize}
		\subsection{Traume}
			\begin{itemize}
				\item
			\end{itemize}
		\subsection{Anafylksi/Allergisk sjokk}
			\begin{itemize}
				\item
			\end{itemize}
	\section{Sutur}
		\begin{itemize}
			\item
		\end{itemize}
	\section{EKG, brystsmerter}
		\begin{itemize}
			\item
		\end{itemize}
	\section{Hjernerystelse/commotio}
		\begin{itemize}
			\item
		\end{itemize}
	\section{Slag/Apoplex}
		\begin{itemize}
			\item
		\end{itemize}
	\section{DVT}
		\begin{itemize}
			\item
		\end{itemize}
	\section{Magesmerter}
		\begin{itemize}
			\item
		\end{itemize}
		\subsection{Barn(under 18 år)}
			\begin{itemize}
				\item
			\end{itemize}
		\subsection{Kvinner 16-40 år}
			\begin{itemize}
				\item
			\end{itemize}
		\subsection{Voksne og eldre}
			\begin{itemize}
				\item
			\end{itemize}
	\section{psykiatri}
		\begin{itemize}
			\item
		\end{itemize}

\newpage
\chapter{Dødsfall}
	\section{Dødsattesten}
		\begin{itemize}
			\item
		\end{itemize}
	\section{Pårørende}
		\begin{itemize}
			\item
		\end{itemize}
	\section{Obduksjon}
		\begin{itemize}
			\item
		\end{itemize}
	\section{Begravelsesbyrå}
		\begin{itemize}
			\item
		\end{itemize}

\newpage
\chapter{Datatrøbbel}
	\section{innloggingsproblemer}
		\subsection{citrix}
			\begin{itemize}
				\item
			\end{itemize}
		\subsection{winmed}
			\begin{itemize}
				\item
			\end{itemize}
	\section{Buypasskort virker ikke}
		\begin{itemize}
			\item
		\end{itemize}
	\section{E resept virker ikke}
		\begin{itemize}
			\item
		\end{itemize}
	\section{Får ikke sendt oppgjør}
		\begin{itemize}
			\item
		\end{itemize}
	\section{sykemelding virker ikke}%Husk hele personnummeret
		\begin{itemize}
			\item
		\end{itemize}

\newpage
\chapter{Når store ulykker skjer}
	\section{Personlige tragedier}
		\begin{itemize}
			\item Snakk med pasienten
			\item Vurder henvising til DPS
			\item Be sykepleier tinge kriseteam dersom det er behov for mer oppfølging
		\end{itemize}
	\section{Mange involverte}
		\begin{itemize}
			\item Ring kommunelegen på \pawmob{} eller \ebmob{}
			\item Gi sykepleier beskjed om å kontakte sin leder for å kalle inn ekstrasykepleiere
			\item Send friske hjem. Avtal dette med sykepleier
			\item Fortsett som lege på legevakten, ikke glem at folk kan få infarkt og annet selv om en stor hendelse skjer
			\item Trykk og hold inne "4" på nødnettradioen, der kan du lytte på hva som skjer
		\end{itemize}

\newpage
\chapter{Vold og trusler}
	\section{Anmeldelse}
		\begin{itemize}
			\item
		\end{itemize}
	\section{Hvis noen er fysisk eller psykisk skadd}
		\begin{itemize}
			\item
		\end{itemize}
	\section{Matrielle skader} 
		\begin{itemize}
			\item
		\end{itemize}
	\section{Muntlige trusler}
		\begin{itemize}
			\item
		\end{itemize}
